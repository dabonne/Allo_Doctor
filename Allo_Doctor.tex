\documentclass{report}

\usepackage[utf8]{inputenc} % un package
\usepackage[T1]{fontenc}      % un second package
\usepackage[francais]{babel}  % un troisième package
\begin{document}
   
\part{Presentation du Cahier des charges}
\chapter{Présentation du projet}
\section{Présentation}
Allo Doctor est un projet innovant initié par un groupe de jeunes informaticien de defferente promotion et de differente experience de programmation. 
\subsection{Présentation des membres du projet}
\begin{itemize}
\item DABONNE Hoda
\item SEGDA Ghyslin
\item SERY Fousseni
\item SODRE Mohamed
\item ZONGO Saoudatou
\end{itemize}
\subsection{Objectifs du projet}
Le projet Allo Doctor est né d'une volonté de doter la population du BURKINA Faso d'un outil informatique à fin reduire les contranites 
du sytème de santé en general et en particulier les difficultés liées a la réalisation examens medicaux sur tout l'etendud du territoire national.
 A terme, ce projet vise la 
mise en place deux applications(Web, et mobile) dans le but de soulager la population dans toute demarche de realisationd'examens medicaux. Problèmes à resoudre:
\begin{itemize}
    \item Suppression des listes d'attente longes,
     interminable, agassante et unitile.
    \item Suppression des allés et retour de l'hopital 
    au laboration d'analyse medical et de la maison 
    au laboration d'analyse medical avec les couts et les retard que cela engendre.
    \item Reduir l'utilisation du papier.
    \item Localiser de façon précise dans le temps et dans l'espace les laboratoires capablent de realiser les analyses medicaux d'un patient
    \item Permettre aux medecin traitant de suivre leur patients en temps reel.
    \item Permettre aux patients de disposer de leur dossier medical en tout lieu et à tout moment.
\end{itemize}    
    % part Prémière partie (end)
\end{document}